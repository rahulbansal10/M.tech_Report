\begin{abstract} 
The goal of unsupervised domain adaptation is to learn a given task for unlabeled target domain data by knowledge transfer from a source domain with rich annotations. Frequency, in real world, “source-domain engineering” becomes a cumbersome process in domain adaptation, since the high-quality source domains highly related to the target domain are hardly available.  Here, we consider a more realistic and  challenging problem setting, wild unsupervised domain adaptation (WUDA), where the source domain data can be noisy, and the model needs to be trained with these noisy labeled data from source domain and unlabeled data from target data. Standard domain adaptation in WUDA setting results in severe negative transfer from noisy source domain. In this work, we  propose a novel network, termed CT-DANN (Co-teaching meets DANN), which incorporates a state-of-the-art approach for handling noisy labels, Co-teaching into standard domain adaptation framework. CT-DANN can be trained using noisy label data and unlabeled target data in an end-to-end manner. Extensive experiments on multiple datasets with different types and levels of noise and comparsion with the state-of-the-art WUDA approach justifies the effectiveness of the proposed framework. 
\end{abstract}